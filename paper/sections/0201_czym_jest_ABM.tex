\section{Czym jest ABM?}
Agent-Based Modeling (ABM) to metoda modelowania systemów złożonych, w której jednostki zwane agentami oddziałują ze sobą oraz z otoczeniem w sposób dynamiczny.
ABM jest szeroko stosowana w badaniach nad zjawiskami społecznymi, ekonomicznymi, biologicznymi, ekologicznymi i innymi systemami złożonymi.

\subsection{Główne cechy ABM}
\begin{itemize}
    \item \textbf{Agenci:} W ABM system składa się z wielu autonomicznych agentów, którzy podejmują decyzje na podstawie zestawu reguł i danych.
          Agenci mogą reprezentować osoby, grupy, organizacje, zwierzęta, komórki biologiczne itp.
    \item \textbf{Interakcje:} Agenci oddziałują ze sobą oraz z otoczeniem, co prowadzi do powstawania złożonych wzorców zachowań na poziomie systemu.
    \item \textbf{Heterogeniczność:} Każdy agent może mieć unikalne cechy, preferencje i zachowania, co umożliwia modelowanie różnorodności występującej w rzeczywistych systemach.
    \item \textbf{Emergencja:} Globalne wzorce zachowań systemu wyłaniają się z interakcji między jednostkowymi agentami, co oznacza, że zachowanie całego systemu nie jest wprost zaprogramowane,
          ale wynika z lokalnych interakcji.
    \item \textbf{Adaptacja:} Agenci mogą uczyć się i dostosowywać swoje zachowania w odpowiedzi na zmieniające się warunki otoczenia lub na podstawie doświadczeń.
\end{itemize}


\subsection{Zastosowania ABM}
\begin{itemize}
    \item \textbf{Ekonomia i finanse:} Analiza zachowań konsumentów, rynków finansowych, decyzji inwestycyjnych oraz dynamiki makroekonomicznej \cite{Tesfatsion2006}.
    \item \textbf{Nauki społeczne:} Modelowanie procesów społecznych, takich jak dyfuzja innowacji, zachowania tłumu, ewolucja norm społecznych oraz migracje \cite{Epstein2007}.
    \item \textbf{Biologia i ekologia:} Badanie dynamiki populacji, interakcji międzygatunkowych, rozprzestrzeniania się chorób oraz zachowań zwierząt \cite{Grimm2005}.
    \item \textbf{Inżynieria i urbanistyka:} Symulacje ruchu drogowego, optymalizacja planowania urbanistycznego, zarządzanie infrastrukturą i systemami transportowymi \cite{Bonabeau2002}.
    \item \textbf{Epidemiologia:} Modelowanie rozprzestrzeniania się chorób zakaźnych, skuteczności interwencji zdrowotnych, oraz dynamiki szczepień \cite{Eubank2004}.
\end{itemize}
