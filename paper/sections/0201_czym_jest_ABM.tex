\section{Czym jest ABM?}
Agent-Based Modeling (ABM) to metoda symulacji, w której jednostki zwane agentami oddziałują ze sobą oraz z otoczeniem według zdefiniowanych reguł \cite{Macal2010}.
\\ Każdy agent może mieć unikalne cechy, preferencje i zachowania, co umożliwia modelowanie różnorodności występującej w rzeczywistych systemach.
\\ Dzięki tym interakcjom można badać złożone zachowania i wzorce, które wyłaniają się na poziomie systemu.
\\ Globalne wzorce zachowań systemu wyłaniają się z interakcji między jednostkowymi agentami, co oznacza, że zachowanie całego systemu nie jest wprost zaprogramowane, ale wynika z lokalnych interakcji.
\\ Agenci mogą uczyć się i dostosowywać swoje zachowania w odpowiedzi na zmieniające się warunki otoczenia lub na podstawie doświadczeń.

ABM ma wiele zastosowań w badaniach nad systemami złożonymi w wielu dziedzinach, takich jak:

\begin{itemize}
    \item \textbf{Ekonomia i finanse:} Analiza zachowań konsumentów, rynków finansowych, decyzji inwestycyjnych oraz dynamiki makroekonomicznej \cite{Tesfatsion2006}.
    \item \textbf{Nauki społeczne:} Modelowanie procesów społecznych, takich jak dyfuzja innowacji, zachowania tłumu, ewolucja norm społecznych oraz migracje \cite{Epstein2007}.
    \item \textbf{Biologia i ekologia:} Badanie dynamiki populacji, interakcji międzygatunkowych, rozprzestrzeniania się chorób oraz zachowań zwierząt \cite{Grimm2005}.
    \item \textbf{Inżynieria i urbanistyka:} Symulacje ruchu drogowego, optymalizacja planowania urbanistycznego, zarządzanie infrastrukturą i systemami transportowymi \cite{Bonabeau2002}.
    \item \textbf{Epidemiologia:} Modelowanie rozprzestrzeniania się chorób zakaźnych, skuteczności interwencji zdrowotnych, oraz dynamiki szczepień \cite{Eubank2004}.
\end{itemize}
