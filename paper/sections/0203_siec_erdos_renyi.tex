\section{Sieć Erdős-Rényi}
Model sieci Erdős-Rényi (ER) jest jednym z najprostszych i najwcześniejszych modeli matematycznych do generowania losowych sieci.
\\ Nazwa pochodzi od nazwisk dwóch matematyków, Paula Erdősa i Alfréda Rényi, którzy w 1959 roku zaproponowali ten model \cite{Erdos1959}.
Model ten jest często stosowany jako punkt odniesienia w analizie sieci i badaniach naukowych dotyczących struktur sieciowych.

\subsection{Definicja modelu Erdős-Rényi}
Istnieją dwie podstawowe wersje modelu Erdős-Rényi:

\begin{itemize}
      \item \textbf{Model $G(n, M)$}: Dany jest zbiór $n$ węzłów, a następnie losowo wybierane jest dokładnie $M$ krawędzi spośród wszystkich możliwych par węzłów.
            \\ Każda z $M$ krawędzi jest dodawana niezależnie.
      \item \textbf{Model $G(n, p)$}: Dany jest zbiór $n$ węzłów, a każda para węzłów jest połączona krawędzią z prawdopodobieństwem $p$, niezależnie od innych par.
            \\ Ostateczna liczba krawędzi w tej wersji jest zmienną losową.
\end{itemize}

Najczęściej stosowaną wersją jest model $G(n, p)$, który jest bardziej intuicyjny i daje większą elastyczność w kontroli gęstości sieci.

\subsection{Właściwości sieci Erdős-Rényi}
\begin{itemize}
      \item \textbf{Rozkład stopni węzłów:} W modelu $G(n, p)$ stopień każdego węzła jest zmienną losową o rozkładzie dwumianowym $\binom{n - 1}{p}$.
            \\ Dla dużych $n$ rozkład ten zbliża się do rozkładu Poissona o wartości \cite{Erdos1960}
            \begin{equation}
                  \lambda = p(n - 1)
                  \label{eq:vertex-degree-distribution}
            \end{equation}
            \newpage
      \item \textbf{Średnia długość najkrótszej ścieżki:} W miarę wzrostu liczby węzłów $n$, średnia odległość między węzłami jest stosunkowo krótka, rzędu $\frac{\ln n}{\ln (np)}$,
            co jest typowe dla tzw. efektu „małego świata” \cite{Newman2003}.
      \item \textbf{Klasteryzacja:} W sieci Erdős-Rényi współczynnik klasteryzacji, czyli prawdopodobieństwo, że dwa węzły sąsiadujące z danym węzłem są również połączone ze sobą,
            jest rzędu $p$ i nie zależy od lokalnych struktur \cite{Watts1998}.
      \item \textbf{Pojawienie się gigantycznego składnika:} W modelu $G(n, p)$ istnieje próg
            \begin{equation}
                  p_c = \frac{1}{n}
                  \label{eq:critical-probability}
            \end{equation}
            powyżej którego zaczyna się tworzyć tzw. gigantyczny składnik – duża, spójna część sieci obejmująca znaczną część węzłów \cite{Bollobas2001}.
      \item \textbf{Losowy charakter:} Sieć nie posiada regularnej struktury, a rozmieszczenie krawędzi jest całkowicie losowe \cite{Erdos1960}.
\end{itemize}

\subsection{Zastosowania sieci Erdős-Rényi}
Model Erdős-Rényi stanowi podstawę teoretyczną do badań nad sieciami losowymi i służy jako punkt odniesienia w analizie bardziej złożonych struktur sieciowych.
Choć rzadko spotyka się go w rzeczywistych sieciach (np. sieciach społecznościowych, biologicznych, komputerowych), pomaga on zrozumieć, jak różnią się rzeczywiste sieci od struktur losowych \cite{Newman2003}.

\subsection{Ograniczenia modelu Erdős-Rényi}
Model Erdős-Rényi nie odzwierciedla dobrze wielu cech rzeczywistych sieci:
\begin{itemize}
      \item \textbf{Mała klasteryzacja:} Rzeczywiste sieci często mają znacznie wyższy współczynnik klasteryzacji niż sieć ER.
            W sieciach rzeczywistych, takich jak sieci społecznościowe lub biologiczne, węzły są bardziej skłonne do tworzenia lokalnych grup (klastrów), czego model ER nie uwzględnia.
            \\ Współczynnik klasteryzacji w modelu ER jest niski i wynosi $\sim \frac{1}{n}$, co nie odpowiada rzeczywistym sieciom \cite{Watts1998}.

      \item \textbf{Rozkład stopni węzłów:} W sieci ER stopnie węzłów mają rozkład zbliżony do Poissona,
            podczas gdy wiele rzeczywistych sieci charakteryzuje się rozkładem potęgowym (np. większość węzłów ma niski stopień, ale istnieje kilka węzłów o bardzo wysokim stopniu).
            Sieci składające się z takich węzłów, tzw. hubów, są typowe dla sieci bezskalowych, jak np. sieci WWW czy sieci społecznościowe, czego model ER nie jest w stanie odwzorować \cite{Barabasi1999}.
\end{itemize}
