\section{Sieć Barabási-Albert}
Model Barabási-Albert (BA) to model sieci bezskalowej, gdzie stopnie węzłów podążają za rozkładem potęgowym.
Model ten został zaproponowany przez Alberta-László Barabásiego i Rékę Alberta w 1999 roku \cite{Barabasi1999} i jest jednym z najważniejszych modeli do opisu sieci rzeczywistych,
takich jak sieci społecznościowe, internet, sieci metaboliczne i wiele innych.

\subsection{Założenia modelu Barabási-Albert}
Model BA opiera się na dwóch kluczowych mechanizmach:

\begin{itemize}
      \item \textbf{Preferencyjne przyłączanie:} Nowe węzły mają większe prawdopodobieństwo połączenia się z węzłami, które już mają wiele połączeń.
            \\ Oznacza to, że „bogaci stają się bogatsi” - węzły o wysokim stopniu przyciągają więcej nowych połączeń.
      \item \textbf{Wzrost:} Sieć rozwija się w czasie, przy czym do istniejącej sieci dodawane są nowe węzły, które tworzą połączenia z już istniejącymi węzłami.
\end{itemize}

\subsection{Algorytm generowania sieci Barabási-Albert}
Proces budowania sieci BA jest następujący:

\begin{enumerate}
      \item \textbf{Inicjalizacja:} Rozpocznij od małej sieci początkowej składającej się z $m_0$ węzłów, które są połączone w pewien sposób.
      \item \textbf{Dodawanie nowego węzła:} Na każdym kroku dodawany jest nowy węzeł, który tworzy $m \leq m_0$ krawędzi łączących go z już istniejącymi węzłami.
      \item \textbf{Preferencyjne przyłączanie:} Prawdopodobieństwo, że nowy węzeł połączy się z istniejącym węzłem $i$, jest proporcjonalne do stopnia węzła $k_i$. Formalnie, prawdopodobieństwo to wynosi:
            \begin{equation}
                  \Pi(k_i) = \frac{k_i}{\sum_{j} k_j}
                  \label{eq:preferential-attachment-probability}
            \end{equation}
            gdzie $k_i$ to stopień węzła $i$, a suma w mianowniku przebiega przez wszystkie istniejące węzły w sieci.
\end{enumerate}

Proces ten jest kontynuowany, aż sieć osiągnie oczekiwaną liczbę węzłów.

\subsection{Właściwości sieci Barabási-Albert}
Model Barabási-Albert opisuje sposób, w jaki sieci rozwijają się zgodnie z zasadą preferencyjnego przyłączania. Model ten ma pewne charakterystyczne właściwości:

\begin{itemize}
      \item \textbf{Preferencyjne przyłączanie:} Nowo dodawane węzły mają większe prawdopodobieństwo połączenia się z węzłami, które już mają dużą liczbę połączeń.
            Oznacza to, że węzły z większą liczbą połączeń (tzw. huby) rosną szybciej, co prowadzi do tworzenia się kilku bardzo dobrze połączonych węzłów w sieci \cite{Barabasi1999}.

      \item \textbf{Rozkład potęgowy stopni węzłów:} Stopnie węzłów w sieci BA (liczba połączeń węzła) podążają za rozkładem potęgowym.
            \\ Oznacza to, że większość węzłów ma niewiele połączeń, ale istnieje niewielka liczba węzłów o bardzo dużej liczbie połączeń.
            Rozkład ten charakteryzuje się rzadkim występowaniem węzłów o bardzo wysokim stopniu, co opisuje wzór:
            \begin{equation}
                  P(k) \sim k^{-\gamma}
                  \label{eq:power-law-probability}
            \end{equation}
            gdzie $P(k)$ jest prawdopodobieństwem, że węzeł ma $k$ połączeń, a $\gamma$ jest parametrem rozkładu, zwykle w przybliżeniu równym 3 \cite{Barabasi1999}.

      \item \textbf{Skalowalność:} Sieci BA są \textit{skalowalne}, co oznacza, że właściwości topologiczne sieci nie zależą od jej wielkości.
            Wzorce łączenia się węzłów pozostają podobne niezależnie od liczby węzłów w sieci, a sieć rozwija się zgodnie z tymi samymi regułami, niezależnie od jej rozmiaru \cite{Newman2003}.

      \item \textbf{Bezskalowość:} Sieci BA są sieciami \textit{bezskalowymi}, co oznacza, że nie ma typowego rozmiaru węzła (stopnia węzła).
            Dominują węzły o małej liczbie połączeń, ale możliwe jest istnienie węzłów o bardzo wysokim stopniu, co różni je od sieci losowych z rozkładem Poissona \cite{Albert2002}.

      \item \textbf{Odporność na uszkodzenia losowe:} Sieci o strukturze BA są stosunkowo odporne na losowe usunięcie węzłów.
            Nawet po usunięciu dużej liczby losowo wybranych węzłów, sieć może nadal funkcjonować, ponieważ większość węzłów ma niewielką liczbę połączeń.
            Jednak sieć jest podatna na ukierunkowane ataki na huby, ponieważ ich usunięcie może prowadzić do fragmentacji sieci \cite{Albert2000}.
\end{itemize}

\subsection{Zastosowania sieci Barabási-Albert}
Model BA dobrze opisuje strukturę wielu rzeczywistych sieci, w tym:
\begin{itemize}
      \item \textbf{Sieci internetowe:} Węzły reprezentują strony internetowe, a krawędzie — linki między nimi. Występuje kilka stron (hubów) mających bardzo dużą liczbę połączeń.
      \item \textbf{Sieci społecznościowe:} Węzły reprezentują osoby, a krawędzie ich relacje społeczne. Niektóre osoby (np. celebryci) mają znacznie więcej połączeń niż inne.
      \item \textbf{Sieci biologiczne:} W sieciach metabolicznych czy sieciach interakcji białek obserwuje się strukturę skali bez charakterystycznej wielkości.

      \item \textbf{Internet i sieci komputerowe:} Internet można modelować jako sieć BA, gdzie węzły reprezentują routery lub serwery, a krawędzie – połączenia między nimi.
            Zasada preferencyjnego przyłączania dobrze odzwierciedla sposób, w jaki nowe witryny internetowe łączą się z popularnymi stronami, tworząc kilka hubów,
            które odpowiadają głównym węzłom sieciowym \cite{Barabasi2003}.

      \item \textbf{Biologia i sieci metaboliczne:} Sieci BA mają zastosowanie w biologii, zwłaszcza w analizie sieci metabolicznych organizmów, gdzie metabolity (węzły) są połączone reakcjami enzymatycznymi (krawędzie).
            Takie sieci również wykazują rozkład potęgowy, z kilkoma metabolitami o bardzo dużej liczbie połączeń, co sprawia, że sieci metaboliczne są odporne na przypadkowe zakłócenia \cite{Jeong2000}.

      \item \textbf{Ekonomia i finanse:} Model BA jest używany do opisu systemów ekonomicznych, takich jak sieci handlowe, gdzie firmy, instytucje finansowe lub rynki są węzłami, a ich wzajemne relacje, jak np. kontrakty, to krawędzie.
            \\ W tych systemach preferencyjne przyłączanie odzwierciedla, jak większe firmy przyciągają więcej partnerów biznesowych, tworząc sieci z rozkładem potęgowym \cite{Battiston2007}.

      \item \textbf{Nauka i sieci cytowań:} W badaniach naukowych model BA jest wykorzystywany do opisu sieci cytowań, gdzie węzłami są artykuły naukowe, a krawędzie oznaczają cytowania między nimi.
            Starsze, bardziej cytowane artykuły są częściej cytowane przez nowe prace, co tworzy strukturę sieci z kilkoma bardzo często cytowanymi artykułami w roli hubów \cite{Price1976}.

      \item \textbf{Media społecznościowe:} Sieci BA są również stosowane w analizie sieci społecznościowych, takich jak Facebook, Twitter czy LinkedIn.
            Użytkownicy tych platform (węzły) tworzą połączenia (krawędzie) z innymi użytkownikami, a zasada preferencyjnego przyłączania powoduje, że popularni użytkownicy (huby) zdobywają jeszcze więcej połączeń,
            co prowadzi do powstawania struktur bezskalowych \cite{Leskovec2008}.
            \newpage

      \item \textbf{Propagacja informacji i epidemie:} Sieci BA znajdują zastosowanie w modelowaniu rozprzestrzeniania się informacji lub chorób w sieciach społecznych.
            Model BA może symulować, jak epidemie lub plotki szybko rozprzestrzeniają się przez kilka kluczowych węzłów (osób o wysokiej liczbie połączeń),
            co pomaga w opracowywaniu strategii kontroli lub przewidywania rozwoju epidemii \cite{Pastor2001}.
\end{itemize}

\subsection{Ograniczenia modelu Barabási-Albert}
Mimo że model BA jest użyteczny, ma pewne ograniczenia:

\begin{itemize}
      \item \textbf{Zbyt uproszczona struktura sieci:} Model BA zakłada, że nowe węzły łączą się tylko na podstawie preferencyjnego przyłączania,
            gdzie prawdopodobieństwo połączenia się z istniejącym węzłem zależy od liczby jego połączeń.
            \\ W rzeczywistych sieciach istnieje jednak wiele innych czynników wpływających na tworzenie połączeń, takich jak geografia, interesy, zasoby, które model BA ignoruje \cite{Newman2003}.

      \item \textbf{Brak heterogeniczności w węzłach:} W modelu BA wszystkie węzły są traktowane jako jednorodne i mają te same właściwości poza liczbą połączeń.
            W rzeczywistości węzły różnią się wieloma cechami, takimi jak zdolności przetwarzania informacji, rola w sieci, czy preferencje, co może wpływać na sposób, w jaki tworzą połączenia \cite{Klemm2002}.

      \item \textbf{Brak uwzględnienia ograniczeń topologicznych:} W modelu BA nie ma ograniczeń dotyczących liczby połączeń, które może mieć węzeł.
            \\ W rzeczywistych sieciach, takich jak sieci społeczne czy biologiczne, istnieją fizyczne lub społeczne ograniczenia na liczbę relacji, które jeden węzeł może utrzymywać, co model BA pomija \cite{Amaral2000}.

      \item \textbf{Zbyt prosty mechanizm wzrostu:} Sieci rzeczywiste rozwijają się często w sposób bardziej złożony, niż sugeruje to model BA.
            Wiele sieci rozwija się nie tylko poprzez przyłączanie nowych węzłów do istniejących, ale także poprzez usuwanie węzłów, zmiany w strukturze połączeń,
            czy tworzenie nowych krawędzi między istniejącymi węzłami, czego model BA nie uwzględnia \cite{Dorogovtsev2002}.

      \item \textbf{Nierealność preferencyjnego przyłączania w pewnych kontekstach:} Zasada preferencyjnego przyłączania, na której opiera się model BA,
            zakłada, że węzły z większą liczbą połączeń zawsze przyciągają więcej nowych połączeń.
            W wielu przypadkach, np. w sieciach społecznych lub ekonomicznych, inne czynniki,
            takie jak reputacja, dostępność lub jakość zasobów, mogą odgrywać większą rolę niż sama liczba połączeń \cite{Boccaletti2006}.

      \item \textbf{Brak dynamiki krawędzi:} W modelu BA nowe węzły dodają nowe krawędzie, ale nie ma możliwości modyfikowania istniejących połączeń ani usuwania starych.
            W rzeczywistych sieciach krawędzie mogą zanikać, np. w wyniku przestarzałych kontaktów, zmian w interesach lub rozwoju technologii \cite{Holme2002}.
\end{itemize}
