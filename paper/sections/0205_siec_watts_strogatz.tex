\section{Sieć Watts-Strogatz}
Model sieci Watts-Strogatz (WS) jest jednym z kluczowych modeli do analizy zjawiska tzw. małego świata w sieciach.
Został zaproponowany przez Duncana Wattsa i Stevena Strogatza w 1998 roku \cite{Watts1998}.
Model ten pozwala generować sieci, które łączą w sobie cechy zarówno sieci regularnych, jak i losowych, co czyni go użytecznym w opisie rzeczywistych sieci społecznych, biologicznych czy technologicznych.

\subsection{Cechy modelu Watts-Strogatz}
Sieci Watts-Strogatz charakteryzują się następującymi właściwościami:
\begin{itemize}
    \item \textbf{Krótka średnia ścieżka:} Węzły są od siebie oddzielone przez stosunkowo małą liczbę połączeń, co jest charakterystyczne dla sieci „małego świata”.
    \item \textbf{Wysoki współczynnik klasteryzacji:} Węzły są silnie połączone z sąsiadami, tworząc lokalne grupy (kliki), co odpowiada klasteryzacji obserwowanej w rzeczywistych sieciach.
    \item \textbf{Efekt małego świata:} Sieć WS stanowi pośrednią strukturę pomiędzy sieciami regularnymi (gdzie węzły są połączone według ustalonego wzorca) a sieciami losowymi (gdzie połączenia są tworzone przypadkowo).
\end{itemize}

\subsection{Algorytm generowania sieci Watts-Strogatz}
Proces generowania sieci WS przebiega następująco:

\begin{enumerate}
    \item \textbf{Konstrukcja pierścienia:} Rozpocznij od utworzenia regularnego pierścienia z $n$ węzłami,
          gdzie każdy węzeł jest połączony z $k/2$ najbliższymi sąsiadami z każdej strony (czyli każdy węzeł ma $k$ połączeń).
    \item \textbf{Przełączanie krawędzi:} Dla każdej krawędzi łączącej węzeł $i$ z węzłem $j$, losowo przełącz ją z prawdopodobieństwem $p$ na nową krawędź,
          łącząc $i$ z losowo wybranym węzłem $m$, pod warunkiem, że nie ma już połączenia między $i$ a $m$.
\end{enumerate}

Parametr $p$ kontroluje stopień losowości w sieci:
\begin{itemize}
    \item Dla $p = 0$ sieć jest całkowicie regularna.
    \item Dla $p = 1$ sieć staje się zupełnie losowa.
    \item Dla wartości $0 < p < 1$ sieć zachowuje zarówno wysoką klasteryzację, jak i krótką średnią ścieżkę, co odpowiada strukturze „małego świata”.
\end{itemize}

\subsection{Właściwości sieci Watts-Strogatz}
\begin{itemize}
    \item \textbf{Współczynnik klasteryzacji:} Dla niewielkich wartości $p$ sieć ma współczynnik klasteryzacji podobny do sieci regularnej.
    \item \textbf{Średnia długość najkrótszej ścieżki:} Nawet dla małych wartości $p$, średnia długość najkrótszej ścieżki w sieci spada gwałtownie, zbliżając się do wartości typowej dla sieci losowej.
    \item \textbf{Mały świat:} Sieci WS mają jednocześnie wysoki współczynnik klasteryzacji oraz krótką średnią długość najkrótszej ścieżki, co stanowi cechę „małego świata”.
\end{itemize}

\subsection{Zastosowania sieci Watts-Strogatz}
Model Watts-Strogatz jest wykorzystywany do analizy i modelowania struktur sieciowych w różnych dziedzinach, takich jak:
\begin{itemize}
    \item \textbf{Sieci społecznościowe:} W sieciach społecznościowych znajomi często tworzą małe, silnie powiązane grupy, ale istnieją także połączenia z osobami spoza tych grup.
    \item \textbf{Sieci biologiczne:} Występuje w neuronowych sieciach mózgowych, gdzie niektóre neurony są bardziej skłonne do połączeń w lokalnych regionach, ale mają także połączenia do odległych obszarów.
    \item \textbf{Sieci komunikacyjne i transportowe:} Takie jak sieci elektryczne czy sieci lotnicze, które mają zarówno krótkie lokalne, jak i długodystansowe połączenia.
\end{itemize}

\subsection{Ograniczenia modelu Watts-Strogatz}
Chociaż model WS odzwierciedla wiele cech rzeczywistych sieci, ma też pewne ograniczenia:
\begin{itemize}
    \item \textbf{Rozkład stopni węzłów:} Sieć WS nie tworzy rozkładu potęgowego stopni węzłów, co ogranicza jej zdolność do odzwierciedlania struktur sieci o charakterze bezskalowym.
    \item \textbf{Brak różnorodności w ewolucji sieci:} Model nie uwzględnia mechanizmu preferencyjnego przyłączania, który jest kluczowy w wielu rzeczywistych sieciach.
\end{itemize}
