\section{Diagram Nolana}
Diagram Nolana jest narzędziem używanym do wizualizacji spektrum politycznego, które rozszerza tradycyjny, jednowymiarowy podział na lewicę i prawicę, wprowadzając dwuwymiarową analizę poglądów politycznych.
Stworzony został przez amerykańskiego libertarianina Davida Nolana w 1970 roku \cite{Nolan1971}.
Diagram ten zyskał popularność wśród osób poszukujących bardziej złożonego sposobu zrozumienia różnorodności ideologicznej, ponieważ uwzględnia zarówno kwestie ekonomiczne, jak i społeczne w analizie politycznej.

\subsection{Kontekst historyczno-naukowy}
Tradycyjne postrzeganie spektrum politycznego jako linii prostej — od skrajnej lewicy do skrajnej prawicy — zostało poddane krytyce przez badaczy, filozofów i działaczy politycznych,
którzy zauważyli, że jednowymiarowy model jest niewystarczający do opisu złożoności ideologii politycznych.
W latach 60. i 70. XX wieku pojawiło się zainteresowanie wielowymiarowym podejściem do analizy politycznej, co skłoniło Davida Nolana do opracowania bardziej złożonego modelu \cite{Nolan1971}.

Nolan zauważył, że różne ideologie mają różne podejścia do kwestii wolności osobistej oraz wolności ekonomicznej.
Jego dwuwymiarowy model pozwalał na bardziej precyzyjną identyfikację pozycji ideologicznych, zwłaszcza dla ideologii, które nie pasowały do tradycyjnej skali lewica-prawica, takich jak libertarianizm.

\subsection{Opis diagramu Nolana}
Diagram Nolana przedstawia spektrum polityczne jako kwadrat podzielony na cztery ćwiartki, które reprezentują różne orientacje polityczne:

\begin{itemize}
      \item \textbf{Oś pozioma (wolność ekonomiczna):} Reprezentuje zakres kontroli państwa nad gospodarką.
            Na lewym końcu znajdują się poglądy opowiadające się za większym wpływem rządu na kwestie ekonomiczne (np. socjalizm),
            natomiast na prawym końcu są poglądy popierające wolny rynek i minimalną interwencję rządową (np. leseferyzm).
      \item \textbf{Oś pionowa (wolność osobista):} Przedstawia zakres swobód obywatelskich i społecznych.
            Na górnym końcu znajdują się ideologie popierające maksymalną wolność osobistą (np. libertarianizm),
            podczas gdy na dolnym końcu znajdują się ideologie popierające większą kontrolę rządową nad życiem osobistym (np. autorytaryzm).
\end{itemize}

Pogranicze tych dwóch osi tworzy cztery główne obszary ideologiczne:
\begin{itemize}
      \item \textbf{Libertarianizm (prawy górny róg):} Wysoka wolność osobista i ekonomiczna.
      \item \textbf{Autorytaryzm (lewy dolny róg):} Niska wolność osobista i ekonomiczna.
      \item \textbf{Lewica (lewy górny róg):} Wysoka wolność osobista, ale niska wolność ekonomiczna.
      \item \textbf{Prawica (prawy dolny róg):} Wysoka wolność ekonomiczna, ale niska wolność osobista.
\end{itemize}

\subsection{Zastosowanie i znaczenie diagramu Nolana}
Diagram Nolana jest używany do analizy i klasyfikacji poglądów politycznych w sposób, który uwzględnia wielowymiarową naturę ideologii. Pomaga on:

\begin{itemize}
      \item \textbf{Rozszerzyć tradycyjne spektrum polityczne:} Pokazuje, że wiele ideologii nie pasuje do jednowymiarowego podziału na lewicę i prawicę,
            zwracając uwagę na to, że kwestie wolności osobistej i ekonomicznej mogą być od siebie niezależne.
      \item \textbf{Identyfikować złożone poglądy polityczne:} Umożliwia dokładniejszą \\ identyfikację pozycji ideologicznych dla jednostek lub organizacji,
            których poglądy są bardziej złożone, niż proste podziały na lewicę i prawicę.
      \item \textbf{Wspierać edukację polityczną:} Jest używany jako narzędzie dydaktyczne do nauczania o różnorodności politycznej, pomagając zrozumieć,
            jak różne ideologie odnoszą się do wolności osobistej i ekonomicznej.
\end{itemize}

\subsection{Krytyka diagramu Nolana}
Mimo swojej popularności i użyteczności, diagram Nolana jest także przedmiotem krytyki:
\begin{itemize}
      \item \textbf{Nadmierne uproszczenie:} Niektórzy badacze argumentują, że nawet model dwuwymiarowy nie oddaje pełni złożoności poglądów politycznych,
            które mogą obejmować wiele innych wymiarów, takich jak ekologia, polityka zagraniczna czy kwestie kulturowe.
      \item \textbf{Subiektywny wybór osi:} Oś wolności ekonomicznej i osobistej może nie być najważniejszym aspektem dla wszystkich osób, co sprawia, że model ten może nie być uniwersalny.
\end{itemize}
