\section{Opis sieci społecznościowej}
Sieć społeczna wymieniona w tytule pracy ma za zadanie odpowiedzieć na pytanie, w jaki sposób formują się przekonania polityczne sieci społecznościowej.
Podstawą realizacji jest agent posiadający specyficzne właściwości.
Współczynnik wpływu na innych wyraża, w jakim stopniu agent wywiera wpływ na opinie sąsiadów, a tym samym na działanie całej sieci.
Elastyczność jednostki określa, w jakim stopniu sąsiedzi agenta są w stanie wpłynąć na zmianę jego opinii.
Opinia początkowa definiuje położenie początkowe agenta na wykresie.

\subsection{Opis działania agenta}
Właściwości agenta mają następujące cechy:
\begin{table}[htbp]
    \centering
    \begin{tabular}{c|c|c}
        \hline
        Zmienna                & Zakres wartości & Rozkład     \\
        \hline
        Wpływ na innych        & 0-1             & równomierny \\
        Elastyczność jednostki & 0,1-1           & beta        \\
        Opinia początkowa      & 0-1             & równomierny \\
    \end{tabular}
    \caption{Parametry agenta}
    \label{tab:agent_parameters}
\end{table}

Ograniczenia wartości mają na celu uniknięcie sytuacji, w której agent mógłby wyjść poza zakres możliwych do uzyskania opinii, tzn. przedział od 0 do 1.
Wpływ na innych jest losowany z rozkładu równomiernego, żeby wyrównać szanse agentów na wpływ na innych.
Opinia początkowa również jest losowana z rozkładu równomiernego, żeby w miarę możliwości równo rozłożyć opinie wszystkich agentów na początku istnienia sieci.
Elastyczność jednostki jest losowana z rozkładu beta, żeby agenci byli bardziej skłonni do zmiany opinii.