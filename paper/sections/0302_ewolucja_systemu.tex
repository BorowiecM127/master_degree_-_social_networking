\section{Aktualizacja opinii agenta}
Ewolucja sieci społecznościowej polega na zmianie opinii agentów, na bazie ich wewnętrznych właściwości oraz połączeń w obrębie sieci.
Pozwalają one na wyznaczenie współczynników, które mają wpływ na środek rozkładu, z którego losowana jest nowa opinia.

Współczynniki użyte do aktualizacji opinii są wymienione poniżej:
\begin{table}[htbp]
    \centering
    \begin{tabular}{c|c}
        \hline
        Zmienna                                            & Zakres wartości \\
        \hline
        Średnia opinii sąsiadów                            & 0-1             \\
        Średnia wpływu sąsiadów                            & 0-1             \\
        Udział znajomych agenta w populacji                & 0-1             \\
        Odległość opinii agenta i średniej opinii sąsiadów & 0-1             \\
        Modyfikator środka rozkładu                        & 0-1             \\
    \end{tabular}
    \caption{Parametry aktualizacji opinii}
    \label{tab:opinion_update_parameters}
\end{table}

Na podstawie powyższych współczynników możemy wyznaczyć nową opinię,
jednak nie wszystkie zostały użyte równocześnie do wyznaczenia wartości pojedynczego modyfikatora.

\subsection{Modyfikatory środka rozkładu}

\subsubsection{Rozkład trójkątny}
Rozkład, z którego jest losowana nowa wartość agenta, jest rozkładem trójkątnym.
Dolną granicę przedziału wyznacza niższa z dwóch wartości: obecna opinia agenta lub średnia opinii sąsiadów.
Górną granicą przedziału jest większa z nich.
Centrum rozkładu $C$ jest wyznaczane wg poniższego wzoru:

$ C = (A_d + N_id) * A_f $,

gdzie:
\begin{itemize}
    \item $A_d$ - stopień agenta (liczba sąsiadów)
    \item $N_id$ - średnia wpływu sąsiadów
    \item $A_f$ - elastyczność agenta
\end{itemize}

\subsubsection{Nowy modyfikator środka rozkładu}
Niestety poprzedni modyfikator środka rozkładu mógł uzyskać wartości powyżej 1, przez stopień agenta, który prowadził do uzyskania przez część agentów wartości większych od 1.
Z tego powodu musiał zostać zmieniony. Nowy wzór modyfikatora środka rozkładu prezentuje się następująco:

$ C = A_f * (N_id + A_nc) / 2 $,

gdzie:
\begin{itemize}
    \item $A_f$ - elastyczność agenta
    \item $N_id$ - średnia wpływu sąsiadów
    \item $A_nc$ - udział sąsiadów agenta w populacji
\end{itemize}

Wartość udziału sąsiadów agenta w populacji wyraża się jako liczba sąsiadów agenta, podzielona przez liczbę wszystkich agentów w populacji.
Aby pokazać, że nowy modyfikator środka rozkładu faktycznie uzyskuje wartości 0-1, przeprowadzony został następujący dowód:
$ C = A_f * (N_id + A_nc) / 2 = [0-1] * ([0-1] + [0-1]) / 2 = [0-1] * [0-1] = [0-1] $

Średnia wpływu sąsiadów będzie średnią ważoną, gdzie wartościami wag będą współczynniki wpływu $A_i$ każdego z nich.