\section{Implementacja sieci społecznej}
W pracy został wykorzystany język Python do implementacji sieci społecznych, zapisu wyników, oraz tworzenia wykresów obrazujących wyniki.
Główną biblioteką wykorzystywaną do implementacji sieci społecznych jest biblioteka NetworkX \cite{NetworkX}.
Biblioteką do tworzenia wykresów została biblioteka Matplotlib \cite{Matplotlib}.
Do odczytu oraz zapisu plików CSV została użyta biblioteka 'CSV' \cite{CSV} ze standardowej biblioteki języka Python.

\subsection{Klasa Network}

Każda sieć społeczna składa się z klasy Graph z biblioteki NetworkX \cite{NetworkX} oraz listy agentów przypisanych do każdego wierzchołka.
Typ aktualizacji opinii jest również zdefiniowany w klasie.
Ponadto, dla celów zapisu wyników, klasa zawiera nazwę sieci społecznej.

\subsection{Klasa Agent}

Każdy agent ma następujące parametry:
\begin{itemize}
    \item Wpływ na innych (influence) - Osiąga wartości 0-1 i określa wartość wpływu na innych agentów
    \item Elastyczność (flexibility) - Osiąga wartości 0,1-1 i określa podatność agenta na zmianę opinii pod wpływem swoich sąsiadów
    \item Opinia (opinion) - Osiąga wartości 0-1 i określa wartość opinii agenta w zależności od opinii sąsiadów
\end{itemize}