\section{Pierwsza symulacja}

\subsection{Aktualizacja opinii agenta}
Ewolucja sieci społecznościowej polega na zmianie opinii agentów, na bazie ich wewnętrznych właściwości oraz połączeń w obrębie sieci.
Pozwalają one na wyznaczenie współczynników, które mają wpływ na środek rozkładu, z którego losowana jest nowa opinia.

Współczynniki użyte do aktualizacji opinii, poza parametrami agenta, są wymienione poniżej:
\begin{table}[htbp]
    \centering
    \begin{tabular}{c|c}
        \hline
        Zmienna                            & Zakres wartości \\
        \hline
        Średnia opinii sąsiadów            & 0-1             \\
        Udział sąsiadów agenta w populacji & 0-1             \\
        Średnia wpływu sąsiadów            & 0-1             \\
    \end{tabular}
    \caption{Parametry aktualizacji opinii}
    \label{tab:opinion_update_parameters}
\end{table}

\subsection{Rozkład trójkątny}
Rozkład, z którego była losowana nowa wartość agenta, jest rozkładem trójkątnym.
Dolną granicę przedziału wyznacza niższa z dwóch wartości: obecna opinia agenta lub średnia opinii sąsiadów, wymieniona w tabeli \ref{tab:opinion_update_parameters}.
Górną granicą przedziału jest większa z nich.
Centrum rozkładu $C$ było wyznaczane wg poniższego wzoru:

\begin{equation}
    C = (A_{ns} + N_{id}) * A_f
    \label{eq:triangular_distribution_center}
\end{equation}

gdzie:
\begin{itemize}
    \item $A_{ns}$ - udział sąsiadów agenta w populacji (w tabeli \ref{tab:opinion_update_parameters})
    \item $N_{id}$ - średnia wpływu sąsiadów (w tabeli \ref{tab:opinion_update_parameters})
    \item $A_f$ - elastyczność agenta
\end{itemize}

Sumowanie średniej wpływu sąsiadów z udziałem sąsiadów agenta w populacji miało za zadanie spowodować, że na agenta z większą ilością sąsiadów wywierany jest większy wpływ.
Z drugiej strony, na odizolowanych osobników wywierany jest mniejszy wpływ, co powoduje, że 'okopują' się oni w swoich poglądach.
Z kolei elastyczność agenta we wzorze pozwala ograniczyć wpływ otoczenia na danego agenta.

Rezultat jest taki, że huby mają duży wpływ na bliskie poglądowo węzły, zacieśniając je coraz bardziej, natomiast węzły z małą liczbą sąsiadów przesuwają się w kierunku huba dużo wolniej.

Aktualizacja współrzędnych modelu była zmieniana w jednej iteracji równocześnie dla wszystkich węzłów w grafie.
Za koniec ewolucji modelu uznajemy moment, w którym opinie wszystkich agentów mieszczą się w kwadracie o boku 0,1.
Symulacje zostały uruchomione dla populacji 20, 50, 100, 200, 500, 1000, 2000 i 5000 agentów.
Wyniki symulacji zostały przedstawione poniżej.

\begin{table}[htbp]
    \centering
    \begin{tabular}{c|c|c|c}
        \hline
        Populacja & \multicolumn{3}{c}{Średnia liczba iteracji}                                    \\
        \hline
                  & Erdős-Rényi                                 & Barabási-Albert & Watts-Strogatz \\
        \hline
        20        & 6,3                                         & 7,2             & 8,1            \\
        50        & 6,3                                         & 9,2             & 11,4           \\
        100       & 6,1                                         & 11,1            & 13,7           \\
        200       & 6,9                                         & 12,1            & 15,5           \\
        500       & 7,7                                         & 13,0            & 17,2           \\
        1000      & 8,1                                         & 14,4            & 19,0           \\
        2000      & 8,7                                         & 14,5            & 19,3           \\
        5000      & 9,6                                         & 15,0            & 19,9           \\
    \end{tabular}
    \caption{Wyniki symulacji wstępnej}
    \label{tab:initial_results}
\end{table}

W tabeli \ref{tab:initial_results} przedstawione zostały wyniki symulacji wstępnej.
Dla większej liczby agentów bardziej widać różnicę pomiędzy szybkością zbiegania agentów do centrum,
jednak nie jest ona zbyt duża, ponieważ spodziewano się liczby około 20 iteracji dla rozmiaru populacji równego 5000.
Ponadto okazało się, że czas zbiegania symulacji rośnie logarytmicznie względem wielkości populacji,
co pozwala przewidzieć czas zbiegania dla danego rozmiaru populacji.


\subsubsection{Nowy modyfikator środka rozkładu}
W trakcie symulacji również kilkukrotnie program zamykał się z błędem.
Okazało się, że wzór \ref{eq:triangular_distribution_center} pozwalał na uzyskanie wartości powyżej 1,
co pokazano poniżej:

\begin{equation}
    C = (A_{ns} + N_{id}) * A_f = ([0-1] + [0-1]) * [0-1] = [0-2] * [0-1] = [0-2]
    \label{eq:triangular_distribution_center_values}
\end{equation}
Z tego powodu musiał zostać zmieniony. Nowy wzór modyfikatora środka rozkładu prezentuje się następująco:

\begin{equation}
    C = \frac{N_{id} + A_{ns}}{2} * A_f
    \label{eq:new_triangular_distribution_center}
\end{equation}

Aby pokazać, że nowy modyfikator środka rozkładu faktycznie uzyskuje wartości 0-1, przeprowadzony został następujący dowód:
\begin{equation}
    C = \frac{[0-1] + [0-1]}{2} * [0-1] = \frac{[0-2]}{2} * [0-1] = [0-1] * [0-1] = [0-1]
    \label{eq:new_triangular_distribution_center_result}
\end{equation}

Po zmianie wzoru aktualizacji opinii uzyskano następujące rezultaty:

\begin{table}[htbp]
    \centering
    \begin{tabular}{c|c|c|c}
        \hline
        Populacja & \multicolumn{3}{c}{Średnia liczba iteracji}                                    \\
        \hline
                  & Erdős-Rényi                                 & Barabási-Albert & Watts-Strogatz \\
        \hline
        20        & 6,1                                         & 7,9             & 7,4            \\
        50        & 5,5                                         & 9,1             & 10,4           \\
        100       & 6,5                                         & 11,2            & 13,7           \\
        200       & 6,9                                         & 11,4            & 15,0           \\
        500       & 7,7                                         & 12,1            & 17,2           \\
        1000      & 7,6                                         & 13,3            & 18,5           \\
        2000      & 8,4                                         & 14,5            & 18,8           \\
        5000      & 8,8                                         & 14,7            & 19,9           \\
    \end{tabular}
    \caption{Wyniki symulacji ze średnią udziału sąsiadów agenta w populacji i średniej opinii sąsiadów}
    \label{tab:results_with_mean_neighbour_population_share_and_neighbour_opinion_mean}
\end{table}

Jak widać w tabeli \ref{tab:results_with_mean_neighbour_population_share_and_neighbour_opinion_mean},
wyniki nie zmieniły się znacząco, natomiast błąd w programie przestał występować. Oznacza to, że należało upodobnić sieć bardziej do sieci rzeczywistej.
