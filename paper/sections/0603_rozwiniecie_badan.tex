\section{Rozwinięcie badań}

% Obraz położenia/gęstości punktów w funkcji numeru iteracji
% Czy zbieganie zależy od liczby osobników w populacji
Poprawa w działaniu symulacji prowadziła do dalszych badań. Należało zrobić obraz gęstości punktów w funkcji numeru iteracji i zbadać, czy zbieżność zależy od liczby osobników w populacji.
% Niech metryką oceny populacji będzie numer iteracji, w której zarówno w jednym, jak i w drugim wymiarze punkty mieszczą się w przedziale o szerokości 0.1.
% Nazwijmy to stabilizacją.
Aby ocenić działanie symulacji, należało obliczyć numer iteracji, w której współrzędne punktów mieszczą się w przedziale o szerokości 0,1, co można uznać za stan stabilizacji.
% Dla każdego rozmiaru populacji wykonajmy po 10 powtórzeń. Jaki jest czas stabilizacji dla rozmiaru populacji 20, 50, 100, 200, 500, 1000, ....
Dla każdego rozmiaru populacji zostały wykonane po 10 powtórzeń. Na tej podstawie obliczono średnią liczbę iteracji prowadzącą do stabilizacji symulacji dla danego rozmiaru populacji.
Okazało się, że czas zbiegania symulacji rośnie logarytmicznie względem wielkości populacji, co pozwala przewidzieć czas zbiegania dla danego rozmiaru populacji.