\section{Drugie poprawki}

Po wykonaniu badań na nowym modelu okazały się konieczne kolejne poprawki.
% 1. Policzyc współrzędne neighbor_opinions nie jako średnią ale jako średnią ważoną z neighbor_influence
Należało policzyć współrzędne opinii sąsiadów jako średnią ważoną z wpływu sąsiadów.
% 2. Testowo - policzyć to samo co do tej pory, ale ze zmienionym neighbor_opinions. Czy choć trochę się opóźni?
Należało sprawdzić, czy zmiana opinii sąsiadów opóźni zbieganie symulacji.
% 3. I to jest kierunek - skoro trójkąt nic nie daje sprawdźmy rozkład wykładniczy. Jak? 
% 3a. Nowy punkt losowany jest na odcinku pomiędzy agent.opinion (ao) i neighbor_opinions (no), przy czym losujemy z rozkładu wykładniczego o średniej zależnej od influence i flexibility. 
%     Oczywiście, im silniejszy wpływ i oporność osobnika tym mniejszy ruch. 
Ponadto, należało zmienić rozkład losowania opinii osobnika, która miała być od teraz zależna od obecnej opinii osobnika i opinii sąsiadów, oraz elastyczności osobnika.
% 3b. Jeśli ciągle punkty dla powiedzmy 50 agentów zbiegają się wprowadzamy współczynnik modyfikujący średnią rozkładu. Czyli mnożymy każdą średnią przez tę samą liczbę. 
%     Schodzimy w dół wg. schematu 0.5, 0.2 0.1, 0.05.
Jeżeli punkty w dalszym ciągu będzie zbiegać szybko, będzie trzeba wprowadzić współczynnik modyfikujący średnią rozkładu.
% 3c. Patrzymy, czy w którymś wreszcie momencie agenci nie zaczną rozdzielać się na idących za dwoma "liderami". Jeśli tak, to mamy coś.
% 3d. Patrzymy na różnicę w zachowaniu dla różnych sieci i różnych wielkości (kiedy pojawia się dążenie do różnych punktów, ile jest tych punktów)
Należy zaobserwować różnicę w zachowaniu dla różnych sieci oraz zależnie od wielkości populacji.
Będzie potrzeba przeanalizowania, czy agenci nie zaczną rozdzielać się na dwie lub więcej grup.

\subsection{Wyniki dla zwykłej średniej i średniej ważonej}

Różnica między wynikami dla zwykłej średniej i średniej ważonej we wzorze aktualizacji opinii ukazują tabele poniżej.

\subsubsection{Sieć Barabasi-Albert}

\begin{table}[htbp]
    \centering
    \begin{tabular}{c|c|c}
        \hline
        Populacja & Średnia zwykła & Średnia ważona \\
        \hline
        20        & 7.1            & 8.9            \\
        50        & 8.4            & 10.8           \\
        100       & 10.9           & 14.8           \\
        200       & 11.0           & 15.8           \\
        500       & 13.5           & 18.2           \\
        1000      & 13.4           & 19.8           \\
        2000      & 14.9           & 20.0           \\
        5000      & 15.2           & 20.0           \\
    \end{tabular}
    \caption{Barabasi-Albert}
    \label{tab:barabasi_albert}
\end{table}

\subsubsection{Sieć Watts-Strogatz}

\begin{table}[htbp]
    \centering
    \begin{tabular}{c|c|c}
        \hline
        Populacja & Średnia zwykła & Średnia ważona \\
        \hline
        20        & 8.6            & 9.1            \\
        50        & 10.8           & 12.3           \\
        100       & 13.3           & 16.9           \\
        200       & 14.9           & 19.2           \\
        500       & 16.6           & 20.0           \\
        1000      & 18.3           & 20.0           \\
        2000      & 19.5           & 20.0           \\
        5000      & 20.0           & 20.0           \\
    \end{tabular}
    \caption{Watts-Strogatz}
    \label{tab:watts_strogatz}
\end{table}

\subsubsection{Sieć Erdos-Renyi}

\begin{table}[htbp]
    \centering
    \begin{tabular}{c|c|c}
        \hline
        Populacja & Średnia zwykła & Średnia ważona \\
        \hline
        20        & 5.5            & 6.6            \\
        50        & 5.7            & 6.0            \\
        100       & 6.4            & 6.2            \\
        200       & 6.2            & 7.1            \\
        500       & 7.6            & 7.5            \\
        1000      & 7.6            & 8.5            \\
        2000      & 8.0            & 8.4            \\
        5000      & 8.5            & 9.6            \\
    \end{tabular}
    \caption{Erdos-Renyi}
    \label{tab:erdos_renyi}
\end{table}