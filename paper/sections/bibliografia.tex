%Rozdział 2.
\bibitem{Tesfatsion2006} Tesfatsion, L., \& Judd, K. L. (Eds.). (2006). \textit{Handbook of Computational Economics: Agent-Based Computational Economics} (Vol. 2). Elsevier.
\bibitem{Epstein2007} Epstein, J. M. (2007). \textit{Generative Social Science: Studies in Agent-Based Computational Modeling}. Princeton University Press.
\bibitem{Grimm2005} Grimm, V., \& Railsback, S. F. (2005). \textit{Individual-based Modeling and Ecology}. Princeton University Press.
\bibitem{Bonabeau2002} Bonabeau, E. (2002). Agent-based modeling: Methods and techniques for simulating human systems. \textit{Proceedings of the National Academy of Sciences}, 99 (Suppl 3), 7280-7287.
\bibitem{Eubank2004} Eubank, S., Guclu, H., Kumar, V. S. A., Marathe, M. V., Srinivasan, A., Toroczkai, Z., \& Wang, N. (2004). Modelling disease outbreaks in realistic urban social networks. \textit{Nature}, 429 (6988), 180-184.

\bibitem{Wasserman1994} Wasserman, S., \& Faust, K. (1994). \textit{Social Network Analysis: Methods and Applications}. Cambridge University Press.
\bibitem{Newman2010} Newman, M. E. J. (2010). \textit{Networks: An Introduction}. Oxford University Press.
\bibitem{Girvan2002} Girvan, M., \& Newman, M. E. J. (2002). Community structure in social and biological networks. \textit{Proceedings of the National Academy of Sciences}, 99 (12), 7821-7826.
\bibitem{Freeman1979} Freeman, L. C. (1979). Centrality in social networks: Conceptual clarification. \textit{Social Networks}, 1(3), 215-239.
\bibitem{Hanneman2005} Hanneman, R. A., \& Riddle, M. (2005). \textit{Introduction to Social Network Methods}. University of California.
\bibitem{Scott2000} Scott, J. (2000). \textit{Social Network Analysis: A Handbook} (2nd ed.). SAGE Publications.
\bibitem{Borgatti1997} Borgatti, S. P., \& Everett, M. G. (1997). Network analysis of two-mode data. \textit{Social Networks}, 19(3), 243-269.
\bibitem{Holme2012} Holme, P., \& Saramäki, J. (2012). Temporal networks. \textit{Physics Reports}, 519(3), 97-125.

\bibitem{Erdos1959} Erdős, P., \& Rényi, A. (1959). On Random Graphs I. \textit{Publicationes Mathematicae}, 6, 290-297.

\bibitem{Barabasi1999} Barabási, A.-L., \& Albert, R. (1999). Emergence of Scaling in Random Networks. \textit{Science}, 286 (5439), 509-512.

\bibitem{Watts1998} Watts, D. J., \& Strogatz, S. H. (1998). Collective dynamics of 'small-world' networks. \textit{Nature}, 393 (6684), 440-442.

\bibitem{Nolan1971} Nolan, D. (1971). The Case for a Libertarian Political Party. \textit{The Individualist}, 1(4), 3-6.

% Rozdział 4.
\bibitem{NetworkX} Hagberg, A. A., Schult, D. A., \& Swart, P. J. (2008). Exploring network structure, dynamics, and function using NetworkX. In \textit{Proceedings of the 7th Python in Science Conference (SciPy2008)} (pp. 11-15). \url{https://networkx.org/}
\bibitem{Matplotlib} Hunter, J. D. (2007). Matplotlib: A 2D graphics environment. \textit{Computing in Science \& Engineering}, 9(3), 90-95. \url{https://matplotlib.org/}
\bibitem{CSV} Python Software Foundation. (n.d.). \textit{csv – CSV File Reading and Writing}. Python Documentation. \url{https://docs.python.org/3/library/csv.html}

% Rozdział 6.
\bibitem{rozklad_trojkatny_wykres} CC BY-SA 3.0, https://commons.wikimedia.org/w/index.php?curid=182090
\bibitem{rozklad_wykladniczy_wykres} Autorstwa Cburnett - Praca własna, CC BY-SA 3.0, https://commons.wikimedia.org/w/index.php?curid=73793