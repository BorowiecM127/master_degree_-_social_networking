\documentclass{wfiisul}
%

\usepackage[utf8]{inputenc}
\usepackage{amsmath}
\usepackage{tabularx}
\usepackage[hidelinks]{hyperref}
\usepackage{afterpage}

%\usepackage{pdflscape}
%\usepackage{afterpage}
%\usepackage{changepage}
%\usepackage{caption}
%\usepackage{rotating} %for sidewisetable
%\usepackage{makecell}
%\usepackage{boldline}
%\usepackage{amsthm}
%\usepackage{amsopn}
%\usepackage{wrapfig}
%\usepackage[table]{xcolor}

\usepackage{framed}
\usepackage{listings}
\usepackage[most]{tcolorbox}
\usepackage{hyphenat}

\hyphenation{pra-cy}

\begin{document}

\tytul{Zastosowanie ABM do analizy procesu formowania opinii w wielowymiarowej przestrzeni opinii}

\autor{Mateusz Borowiec}
\nralbumu{382765}

\promotor{Dr hab. Tomasz Gwizdałła, prof. UŁ}
\katedra{Systemów Inteligentnych}

\kierunek{informatyka}

\specjalnosc{informatyka stosowana}
\typpracy{magisterska}
\sciezka{Sztuczna inteligencja}

\stronatytulowa


\chapter{Wstęp}

\chapter{Podstawy teoretyczne}

TYMCZASOWY TEKST Z OPISU PRACY \\
Metody agentowe (Agent Based Modelling) są jedną z popularnych metod analizy wielu procesów zachodzących w społecznościach.
Jednym z takich procesów jest rozprzestrzenianie się opinii, przy czym opinia może być reprezentowana w różny sposób.
W prezentowanej pracy ma ona być przedstawiona w formie położenia w znormalizowanej przestrzeni wielowymiarowej (przykładem takiej przestrzeni jest dwuwymiarowy diagram Nolana).
Społeczność zostanie przedstawiona w formie typowych grafów społecznościowych (BA, WS, ER).
Celem pracy jest określenie stanów końcowych takich modeli dla wybranych funkcji modyfikacji opinii oraz czasów dojścia do tych stanów.
Wśród pytań, które pojawiają się w trakcie rozwiązywania takiego problemu są takie jak: 
pytanie o istnienie (dla danej funkcji modyfikacji) krytycznej wielkości próbki, dla której struktura rozwiązania ulega zmianie (np. pojawiają się odstępstwa od jednomyślności) czy pytanie o możliwość włączenia czynników zewnętrznych.

\chapter{Opis metod (algorytmy, założenia, warunki graniczne)}

Agent posiada następujące parametry:
\begin{table}[htbp]
  \centering
  \begin{tabular}{c|c|c}
    \hline
    Zmienna & Zakres wartości & Rozkład \\
    \hline
    Wpływ na innych & 0-1 & równomierny \\
    Elastyczność jednostki & 0,1-1 & beta \\
    Opinia początkowa & 0-1 & równomierny \\
  \end{tabular}
  \caption{Parametry agenta}
  \label{tab:agent_parameters}
\end{table}

Aktualizacja opinii składa się z następujących zmiennych:
\begin{table}[htbp]
  \centering
  \begin{tabular}{c|c}
    \hline
    Zmienna & Zakres wartości \\
    \hline
    Średnia opinii sąsiadów & 0-1 \\
    Srednia wpływu sąsiadów & 0-1 \\
    Udział znajomych agenta w populacji & 0-1 \\
    Odległość opinii agenta i średniej znajomych & 0-1 \\
    Modyfikator środka rozkładu & Elastyczność agenta * (udział znajomych agenta w populacji + wpływ sąsiadów) = [0-1] * ([0-1] + [0-1]) \\
  \end{tabular}
  \caption{Parametry aktualizacji opinii}
  \label{tab:opinion_update_parameters}
\end{table}

\section{Rozkład trójkątny}

Opinia sąsiadów: \\
- średnia opinii sąsiadów: 0-1 \\
- średnia wpływu sąsiadów: 0-1 \\

Centrum rozkładu trójkątnego: (stopień jednostki + średnia wpływu sąsiadów) * elastyczność jednostki \\
Minimum rozkładu: opinia jednostki \\
Maksimum rozkładu: średnia opinii sąsiadów \\
Odległość między opiniami = abs(opinia jednostki - średnia opinii sąsiadów) \\

\section{Nowy modyfikator środka rozkładu}

Elastyczność agenta * średnia([udział znajomości agenta w populacji, wpływ sąsiadów]) = [0-1] * ([0-1] * [0-1] / 2) = [0-1] * [0-1] = [0-1]. \\
Średnia wpływu sąsiadów będzie średnią ważoną. \\

\section{Odpowiedzi na pytania - kartka nr 3}

\chapter{Opis technologii wykorzystanych w pracy}

W pracy został wykorzystany język Python do implementacji zarówno sieci społecznych, zapisu wyników, jak i wykresów obrazujących wyniki. \\
Główną biblioteką wykorzystywaną do implementacji sieci społecznych jest biblioteka NetworkX. \\
Biblioteką do tworzenia wykresów została biblioteka Matplotlib. \\
Do odczytu / zapisu plików CSV została użyta biblioteka `csv`. \\

\chapter{Opis implementacji}

\section{Klasa Network}

Każda sieć społeczna składa się z grafu NetworkX oraz listy agentów, przypisanych do każdego wierzchołka. \\
Typ aktualizacji opinii jest również zdefiniowany w klasie. \\
Ponadto, do celów logowania, klasa zawiera nazwę sieci społecznej. \\

\section{Klasa Agent}

Każdy agent ma następujące parametry:
\begin{itemize}
  \item Wpływ na innych (influence) - Ten parametr osiąga wartości 0-1 i określa wartość wpływu na innych agentów
  \item Elastyczność (flexibility) - Osiąga wartości 0-1 i określa podatność agenta na zmianę opinii pod wpływem swoich sąsiadów
  \item Opinia (opinion) - Osiąga wartości 0-1 i określa wartość opinii agenta w zależności od opinii sąsiadów
\end{itemize}

\section{Implementacja części algorytmicznej}
\section{Dokumentacja projektu jak dla prac inżynierskich (jeśli potrzeba)}
\chapter{Opis wykonanych w ramach pracy badań, symulacji, eksperymentów}
\chapter{Analiza otrzymanych wyników}
\chapter{Podsumowanie}

\listoftables
\listoffigures

\thebibliography{99}

\end{document}
